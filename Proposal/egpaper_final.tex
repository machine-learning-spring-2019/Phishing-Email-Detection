\documentclass[10pt,twocolumn,letterpaper]{article}

\usepackage{cvpr}
\usepackage{times}
\usepackage{epsfig}
\usepackage{graphicx}
\usepackage{amsmath}
\usepackage{amssymb}

\usepackage{enumitem}
\usepackage{blindtext}
\usepackage{enumitem}
\usepackage{xcolor}
\setlist[description]{font=\textendash\enskip\scshape\bfseries}

% Include other packages here, before hyperref.

% If you comment hyperref and then uncomment it, you should delete
% egpaper.aux before re-running latex.  (Or just hit 'q' on the first latex
% run, let it finish, and you should be clear).
\usepackage[breaklinks=true,bookmarks=false]{hyperref}

\cvprfinalcopy % *** Uncomment this line for the final submission

\def\cvprPaperID{****} % *** Enter the CVPR Paper ID here
\def\httilde{\mbox{\tt\raisebox{-.5ex}{\symbol{126}}}}

% Pages are numbered in submission mode, and unnumbered in camera-ready
%\ifcvprfinal\pagestyle{empty}\fi
\setcounter{page}{1}
\begin{document}

%%%%%%%%% TITLE
\title{A Machine Learning Approach to Phishing Detection}

\author{Kendra Maggiore\\
Texas State University\\
%Institution1 address\\
{\tt\small firstauthor@i1.org}
% For a paper whose authors are all at the same institution,
% omit the following lines up until the closing ``}''.
% Additional authors and addresses can be added with ``\and'',
% just like the second author.
% To save space, use either the email address or home page, not both
\and
Jon Pugh\\
Texas State University\\
%First line of institution2 address\\
{\tt\small JPugh90@gmail.com}
\and
James Knepper\\
Texas State University\\
%First line of institution2 address\\
{\tt\small jbk30@txstate.edu}
}

\maketitle
%\thispagestyle{empty}

%%%%%%%%% ABSTRACT
\begin{abstract}
   With the evolution of email as one of the primary means of communication, certain problems arise with exploitation such as Phishing.  This project will deal with the application of machine learning algorithms to detect phishing attacks.  The algorithms we will use are logistical regression and support vector machine.  These algorithms will be trained and tested on large data sets, and the results of such applications will be compared for accuracy and efficiency.  
\end{abstract}

%%%%%%%%% BODY TEXT
\section{Introduction}

Email is a widely used and effective means of electronic communication, and it is a commonly accepted method of contact between entities regardless of the specific field.  Since email is such a primary correspondence in the modern world, there are those who wish to exploit email service and its users.  A common exploitation technique of email services is known as Phishing.  Phishing is a major threat to today’s email communication.  In a Phishing attempt, a malicious entity will send an email to a user that appears to be legitimate.  The purpose of the email is to solicit the user to provide personal or sensitive information, often by containing a URL to an unsecure or malicious website.


\begin{figure*}
\begin{center}
\fbox{\rule{0pt}{2in} \rule{.9\linewidth}{0pt}}
\end{center}
   \caption{Example of a short caption, which should be centered.}
\label{fig:short}
\end{figure*}

%------------------------------------------------------------------------
\section{Existing Solutions}

~\cite{diego} is an example of a phishing email detection machine learning algorithm that utilizes Rapid Miner. We plan to use a similar feature set but we will implement our algorithm with Python to provide greater flexibility in development. It also uses a dataset that was created in 2007. We will use a dataset available from ~\cite{monkey} that was created in 2018. If time allows we will implement TF-IDF as discussed in ~\cite{CENreport} as an additional feature.


%------------------------------------------------------------------------
\section{Preliminary Plan}

\begin{description}
\item Convert latest .mbox from~\cite{monkey} to a .csv containing \\feature set
\item Design and implement Logistic Regression algorithm
\item Design and implement SVM algorithm    
\item Implement TF-IDF as an additional feature
\end{description}

\section{References}

List and number all bibliographical references in 9-point Times,
single-spaced, at the end of your paper. When referenced in the text,
enclose the citation number in square brackets, for
example~\cite{CENreport}.  Where appropriate, include the name(s) of
editors of referenced books.

\begin{table}
\begin{center}
\begin{tabular}{|l|c|}
\hline
Method & Frobnability \\
\hline\hline
Theirs & Frumpy \\
Yours & Frobbly \\
Ours & Makes one's heart Frob\\
\hline
\end{tabular}
\end{center}
\caption{Results.   Ours is better.}
\end{table}


{\small
\bibliographystyle{ieee}
\bibliography{egbib}
}

\end{document}
